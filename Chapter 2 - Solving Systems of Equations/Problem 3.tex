\documentclass{article}

\usepackage{graphicx}
\usepackage{subcaption}
\usepackage{float}
\usepackage{enumitem}
\usepackage{amsmath}
\usepackage{amsthm}
\usepackage{booktabs}
\usepackage{amssymb}
\usepackage{geometry}
\usepackage{mathrsfs}
\usepackage{centernot}
\renewcommand{\qedsymbol}{$\blacksquare$}

\geometry{letterpaper, margin = 1in}

\begin{document}
	
	\begin{flushleft}	
	
		Brandon M. Keltz\\
		An Introduction to Computational Science by Allen Holder and Joseph Eichholz\\
		Chapter 2 - Solving Systems of Equations\\
		November 18, 2019\\\
		
		\textit{Problem 3}. Let $Q \succ 0$ be an $n \times n$ matrix and let $v_1, v_2, \ldots, v_n$ be a collection of $Q$-conjugate $n$-vectors. Show that the collection is linearly independent. \\\
		
		\begin{proof}
		
			Consider the linear combination of the $Q$-conjugate $n$-vectors such that
			\begin{align*}
				\alpha_1 v^1 + \alpha_2 v^2 + \ldots + \alpha_n v^n = 0,
			\end{align*}
			where $\alpha_i \in \mathbb{R}$ for all $i \in \left\{ 1, 2, \ldots, n \right\}$. Since $Q \succ 0$
			\begin{align*}
				\alpha_1 v^1 + \alpha_2 v^2 + \ldots + \alpha_n v^n = 0 \iff \alpha_1 Q v^1 + \alpha_2 Q v^2 + \ldots + \alpha_n Q v^n = 0.
			\end{align*}
			Consider an arbitrary $v^j \in \left\{ v^1, v^2, \ldots, v^n \right\}$. We now have 
			\begin{align*}
				\alpha_1 Q v^1 + \alpha_2 Q v^2 + \ldots + \alpha_n Q v^n = 0 \iff \alpha_1 \left( v^i \right)^T Q v^1 + \alpha_2, \left( v^i \right)^T Q v^2 + \ldots + \alpha_n \left( v^i \right)^T Q v^n = 0,
			\end{align*}
			which gives for $i \neq j$
			\begin{align*}
				\alpha_j \left( v^i \right)^T Q v^j = 0
			\end{align*}
			because every vector $v^i \in \left\{ v^1, v^2, \ldots, v^n \right\}$ is $Q$-conjugate. Notice for $i = j$ we have
			\begin{align*}
				\alpha_i \left( v^i \right)^T Q v^i = 0 \iff \alpha_i ||v^i||_Q = 0 \implies \alpha_i = 0.
			\end{align*}
			Since $\alpha_i$ is arbitrary, then this is applied to every vector in the collection. Thus, $\alpha_i = 0$ for all $i \in \left\{ 1, 2, \ldots, n \right\}$, which means the collection must be linear independent. 
		
		\end{proof}
	
	\end{flushleft}
	
\end{document}