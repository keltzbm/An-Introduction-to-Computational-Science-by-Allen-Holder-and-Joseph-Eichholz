\documentclass{article}

\usepackage{graphicx}
\usepackage{subcaption}
\usepackage{float}
\usepackage{enumitem}
\usepackage{amsmath}
\usepackage{amsthm}
\usepackage{booktabs}
\usepackage{amssymb}
\usepackage{geometry}
\usepackage{mathrsfs}
\usepackage{centernot}
\renewcommand{\qedsymbol}{$\blacksquare$}

\geometry{letterpaper, margin = 1in}

\begin{document}
	
	\begin{flushleft}	
	
		Brandon M. Keltz\\
		An Introduction to Computational Science by Allen Holder and Joseph Eichholz\\
		Chapter 2 - Solving Systems of Equations\\
		November 15, 2019\\\
		
		\textit{Problem 2}. Show that the product of upper (lower) triangular matrices is upper (lower) triangular. \\\
		
		\begin{proof}
		
			Without loss of generality, let $A$ and $B$ be upper triangular matrices, where $A$ and $B$ have dimensions $m \times n$ and $n \times m$, respectively. The product of $A$ and $B$ is
			\begin{align*}
				c_{ij} = \sum_{k = 1}^{\min \{ m, n \}} a_{ik} b_{kj} = \sum_{k = 1}^{i - 1} a_{ik} b_{kj} + \sum_{k = i}^j a_{ik} b_{jk} + \sum_{k = j + 1}^{\min \{ m, n \}} a_{ik} b_{kj}.
			\end{align*}
			Notice that for all $k < i$ we have
			\begin{align*}
				\sum_{k = 1}^{i - 1} a_{ik} b_{kj} = 0
			\end{align*}
			because $A$ is upper triangular. Similarly, for all $k > j$ we have
			\begin{align*}
				\sum_{k = j + 1}^{\min \{ m, n \}} a_{ik} b_{kj} = 0
			\end{align*}
			because $B$ is upper triangular. This gives 
			\begin{align*}
				c_{ij} = \sum_{k = i}^j a_{ik} b_{jk},
			\end{align*}
			which is zero for $j < i$. So, by definition $C$ must also be an upper triangular matrix.			
		
		\end{proof}
	
	\end{flushleft}
	
\end{document}